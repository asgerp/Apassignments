%
%  untitled
%
%  Created by Asger Pedersen on 2011-10-10.
%  Copyright (c) 2011 . All rights reserved.
%
\documentclass[]{article}

% Use utf-8 encoding for foreign characters
\usepackage[utf8]{inputenc}

% Setup for fullpage use
\usepackage{fullpage}

% Uncomment some of the following if you use the features
%
% Running Headers and footers
%\usepackage{fancyhdr}

% Multipart figures
%\usepackage{subfigure}

% More symbols
%\usepackage{amsmath}
\usepackage{amssymb}
%\usepackage{latexsym}

% Surround parts of graphics with box
\usepackage{boxedminipage}

% Package for including code in the document
\usepackage{listings}

% If you want to generate a toc for each chapter (use with book)
\usepackage{minitoc}

% This is now the recommended way for checking for PDFLaTeX:
\usepackage{ifpdf}

%\newif\ifpdf
%\ifx\pdfoutput\undefined
%\pdffalse % we are not running PDFLaTeX
%\else
%\pdfoutput=1 % we are running PDFLaTeX
%\pdftrue
%\fi

\ifpdf
\usepackage[pdftex]{graphicx}
\else
\usepackage{graphicx}
\fi
\title{Number sets in Prolog}
\author{ Asger Pedersen And Kristoffer Cobley}
\setlength{\parindent}{0pt}
\setlength{\parskip}{2ex}
\linespread{1.3}

\begin{document}

\ifpdf
\DeclareGraphicsExtensions{.pdf, .jpg, .tif}
\else
\DeclareGraphicsExtensions{.eps, .jpg}
\fi

\maketitle
\setcounter{tocdepth}{1}
\tableofcontents
\newpage
\section{Introduction} % (fold)
\label{sec:introduction}
In the following sections we will give a short account of our implementation of the number set predicates in Prolog. After that we will explain how we tested the predicates and lastly give an overall assesment of our implementation.
% section introduction (end)
\section{Implementation}
\subsection{Less} % (fold)
\label{sec:less}
To implement less\textbackslash2 we had to implement a helper predicate, getpred\textbackslash2, that gives the predecessor of a natural number, getpred(s(X),X). Less\textbackslash2 then works by "substracting" 1 from each of the terms and calling less recursively on the new terms. To stop the recursion we have a base fact, less(z,s(X)). That ensures that less(z,z) would fail. Calling less(s(x),z) would also fail because it would not be matched by getpred\textbackslash2.
% subsection less (end)
\subsection{Checkset} % (fold)
\label{sec:checkset}
To implement checkset\textbackslash1 we identified 2 cases and defined a helper predicate succ\textbackslash1. succ\textbackslash1 returns true if succ(z) is called, and will call it self recursively with succ(X) if it is called with s(X). All other cases fail. The cases found for checkset\textbackslash1 are:
\begin{itemize}
	\item The set with only one element, X. Checkset\textbackslash1 should succeed if X is a natural number, to ensure this succ\textbackslash1 has to evaluate to true.
	
	\item The set with strictly more than one elements, $[X,Y|Z]$. Checkset\textbackslash1 should succeed if less(X,Y) and checkset($[Y|Z]$) are true. This via transitivity ensures that the list is ordered and less also  ensures that the elements are natural numbers.
\end{itemize}	

% subsection checkset (end)
\subsection{Ismember} % (fold)
\label{sec:ismember}
To implement ismember\textbackslash3 we identified 4 cases:
\begin{itemize}
	\item The set we are asking about is empty, therefore X can't be a member of the set. $t_3$ is "no".
	\item X is in the head of the set. $t_3$ is "yes", because as the set is ordered we know that X is a member and we don't have to search further.
	\item X is bigger than the head of the set. We know that we have not passed X, but we can't be certain if it will be in the set so we call ismember\textbackslash3 recursively on the tail of the set.
	\item X is less than the head of the set. Because the set is ordered we know X is not in the set. $t_3$ is "no".
\end{itemize}
The query ismember(N, [s(z),s(s(s(z)))],A) should return all the solutions to the queries ismember(N,[s(z),s(s(s(z)))],yes) and ismember(N,[s(z),s(s(s(z)))],no). Our implementation returns the solutions for yes first, then returns the solutions for no. 
% subsection ismember (end)
\subsection{Union} % (fold)
\label{sec:union}

To implement union\textbackslash3, we first had to identify the possible cases. The following are the cases we have found to hold, given that sets X,Y are ordered, correct natural number sets:

\begin{itemize}
	\item the union of two empty sets is the empty set.
	
	\item the union of a non-empty set X with an empty set must mean the head of X is in the result set. The rest of the solution is found by tail recursion on the remaining Xs. This can be seen from the definition of $X \cup \emptyset$.
	
	\item the union of an empty set X with a non-empty set Y must mean the head of Y is in the result set. The rest of the solution is found by tail recursion on the remaining Ys. This can be seen from the definition of $Y \cup \emptyset$.
	
	\item If the head of non-empty sets X,Y is the same, the head must be in the result set. The rest of the solution is found by tail recursion on the remaining Xs and Ys. Again, this follows from the definition of $X \cup Y$.
	
	\item If the head of non-empty set X differs from the head of non-empty set Y, and the head of X is less than the head of Y, then the head of X must be in the result set at this point. The rest of the solution is found by tail recursion on the remaining Xs and Y$|$Ys. This also follows from the definition of $X \cup Y$. Head X is chosen here as it must be inserted here for the result set to be ordered. 
	
	\item If the head of non-empty set Y differs from the head of non-empty set X, and the head of Y is less than the head of X, then the head of Y must be in the result set at this point. The rest of the solution is found by tail recursion on the remaining X$|$Xs and Ys. This also follows from the definition of $X \cup Y$. Head Y is chosen here as it must be inserted here for the result set to be ordered.  

\end{itemize}

It can be seen that this solution will terminate as each callee must necessarily work on a smaller subproblem.\par   

% subsection union (end)
\subsection{Intersection} % (fold)
\label{sec:intersection}

To implement intersection\textbackslash3, we first had to identify all the possible cases. The following are the cases we have found to hold, given that sets X,Y are ordered, correct natural number sets:

\begin{itemize}
	\item the intersection of two empty sets is the empty set. This follows from the definition of $\emptyset \cap \emptyset$.
	\item if the head of non-empty set X differs from the head of non-empty set Y, and head X is less than head Y, then head X can't possibly be in the result set. We discard head X and recursively call with arguments Xs, Y$|$Ys. This follows from the definition of $X \cap Y$. We can safely discard head X as sets X,Y are ordered.
	\item if the head of non-empty set Y differs from the head of non-empty set X, and head Y is less than head X, then head Y can't possibly be in the result set. We discard head Y and recursively call with arguments X$|$Xs, Ys. This also follows from the definition of $X \cap Y$. Again, we can safely discard head Y as sets X,Y are ordered.
	\item the intersection of a non-empty set X with the empty set Y means the head of X can't possibly be in the result set. In this case we discard the head of X and recursively call with the tail of X. This follows from the definition of $X \cap \emptyset$.
	\item the intersection of a non-empty set Y with the empty set X means the head of Y can't possibly be in the result set. In this case we discard the head of Y and recursively call with the tail of Y. This follows from the definition of $Y \cap \emptyset$.

\end{itemize}

It can be seen that this solution will terminate as each callee must necessarily work on a smaller subproblem.\par

% subsection intersection (end)
\section{Tests} % (fold)
\label{sec:tests}
We chose to use PLunit to test our solution. Our testing strategy consisted of identifying all the cases in a given predicate and the write tests that covered them all. Mostly this ment all the cases where the predicate should evaluate to true and then the cases where the predicate should fail. We also included the tests suggested in the assignment. \\
We wrote tests for each predicate and included the tests in the file where we implemented the solution. \\

The test are pretty straight forward but some are some what interesting. The test "less\_var2" checks that our predicate finds correct answers if we ask less(s(s(x)),X). To get something measurable we had to alter the query a little. As the query would give the answer s(s(s(\_GXXX))) we bound X by also asking less(X, s(s(s(s(z)))))). It is also interesting to note that we were not able to use the exact same tactic while testing the query ismember(N,[s(z),s(s(s(z)))],Z), but had to do it in two steps that each checks for what values Z and N take. In testing union we made sure to test for uninstantiated variables for $t_1$, $t_2$ and $t_3$. As we have not implemented this for intersect we did not test this, safe for $t_3$.\\
We constructed 55 tests that all yield the expected result.

% section tests (end)
\section{Assesment} % (fold)
\label{sec:assesment}
As the assignment stated, our program consist only of pure rules and facts. We have tried to eliminate redundant rules and facts, and as far as we can see we have none, but we might have missed some.
None of our predicates enter into infinite loops and all of the them succeed only once.
For every predicate we have attempted to write rules based on similar mathematical definitions. This and our thorough testing leads us to believe we have achieved in solving the exercise well.\par

% section assesment (end)
\bibliographystyle{plain}
\bibliography{}
\end{document}