%
%  untitled
%
%  Created by Asger Pedersen on 2011-10-10.
%  Copyright (c) 2011 . All rights reserved.
%
\documentclass[]{article}

% Use utf-8 encoding for foreign characters
\usepackage[utf8]{inputenc}

% Setup for fullpage use
\usepackage{fullpage}

% Uncomment some of the following if you use the features
%
% Running Headers and footers
%\usepackage{fancyhdr}

% Multipart figures
%\usepackage{subfigure}

% More symbols
%\usepackage{amsmath}
\usepackage{amssymb}
%\usepackage{latexsym}

% Surround parts of graphics with box
\usepackage{boxedminipage}

% Package for including code in the document
\usepackage{listings}

% If you want to generate a toc for each chapter (use with book)
\usepackage{minitoc}

% This is now the recommended way for checking for PDFLaTeX:
\usepackage{ifpdf}

%\newif\ifpdf
%\ifx\pdfoutput\undefined
%\pdffalse % we are not running PDFLaTeX
%\else
%\pdfoutput=1 % we are running PDFLaTeX
%\pdftrue
%\fi

\ifpdf
\usepackage[pdftex]{graphicx}
\else
\usepackage{graphicx}
\fi
\title{Number sets in Prolog}
\author{ Asger Pedersen And Kristoffer Cobley}
\setlength{\parindent}{0pt}
\setlength{\parskip}{2ex}
\linespread{1.3}

\begin{document}

\ifpdf
\DeclareGraphicsExtensions{.pdf, .jpg, .tif}
\else
\DeclareGraphicsExtensions{.eps, .jpg}
\fi

\maketitle
\setcounter{tocdepth}{1}
\tableofcontents
\newpage
\section{Introduction} % (fold)
\label{sec:introduction}
In the following sections we will give a short account of the implementation of the number set predicates in Prolog. After that we will explain how we tested the predicates and lastly give an overall assesment of our implementation.
% section introduction (end)
\section{Less} % (fold)
\label{sec:less}
To implement less\textbackslash2 we had to implement a helper predicate, getpred\textbackslash2, that gives the predecessor of a natural number, getpred(s(X),X). Less\textbackslash2 then works by "substracting" 1 from each of the terms and calling less recursively on the new terms. To stop the recursion we have a base fact, less(z,s(X)). That ensures that less(z,z) would fail. Calling less(s(x),z) would also fail because it would not be matched by getpred\textbackslash2.
% section less (end)
\section{Checkset} % (fold)
\label{sec:checkset}
To cases, only one element in set, Check that that element is natural number, else check that the head of the set is smaller than the next element and call checkset recursively on the tail. transitivity ensures x < y, y < z => x < z.
% section checkset (end)
\section{Ismember} % (fold)
\label{sec:ismember}
Four cases, X is member of empty set. Z = no,
X is head of the set, Z = yes.
X is bigger than  head in the set, call ismember recursively on tail.
X is smaller than head of set. X cant be in the set, Z = no.
% section ismember (end)
\section{Union} % (fold)
\label{sec:union}

To implement union\textbackslash3, we first had to identify the possible cases. The following are the cases we have found to hold, given that sets X,Y are ordered, correct natural number sets:

\begin{itemize}
	\item the union of two empty sets is the empty set.
	\item the union of a non-empty set X with an empty set must mean the head of X is in the result set. The rest of the solution is found by tail recursion on the remaining Xs.
	\item the union of an empty set X with a non-empty set Y must mean the head of Y is in the result set. The rest of the solution is found by tail recursion on the remaining Ys.
	\item If the head of non-empty sets X,Y is the same, the head must be in the result set. The rest of the solution is found by tail recursion on the remaining Xs and Ys.
	\item If the head of non-empty set X differs from the head of non-empty set Y, and the head of X is less than the head of Y, then the head of X must be in the result set at this point. The rest of the solution is found by tail recursion on the remaining Xs and Y$|$Ys.
	\item If the head of non-empty set Y differs from the head of non-empty set X, and the head of Y is less than the head of X, then the head of Y must be in the result set at this point. The rest of the solution is found by tail recursion on the remaining X$|$Xs and Ys.  

\end{itemize}

It can be seen that this solution will terminate as each callee must necessarily work on a smaller subproblem.\par   

% section union (end)
\section{Intersection} % (fold)
\label{sec:intersection}

To implement intersection\textbackslash3, we first had to identify all the possible cases. The following are the cases we have found to hold, given that sets X,Y are ordered, correct natural number sets:

\begin{itemize}
	\item the intersection of two empty sets is the empty set.
	\item if the head of non-empty set X differs from the head of non-empty set Y, and head X is less than head Y, then head X can't possibly be in the result set. We discard head X and recursively call with arguments Xs, Y$|$Ys.
	\item if the head of non-empty set Y differs from the head of non-empty set X, and head Y is less than head X, then head Y can't possibly be in the result set. We discard head Y and recursively call with arguments X$|$Xs, Ys.
	\item the intersection of a non-empty set X with the empty set Y means the head of X can't possibly be in the result set. In this case we discard the head of X and recursively call with the tail of X.
	\item the intersection of a non-empty set Y with the empty set X means the head of Y can't possibly be in the result set. In this case we discard the head of Y and recursively call with the tail of Y.

\end{itemize}

It can be seen that this solution will terminate as each callee must necessarily work on a smaller subproblem.\par

% section intersection (end)
\section{Tests} % (fold)
\label{sec:tests}

% section tests (end)
\section{Assesment} % (fold)
\label{sec:assesment}

% section assesment (end)
\bibliographystyle{plain}
\bibliography{}
\end{document}